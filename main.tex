\documentclass{article}
\usepackage[utf8]{inputenc}

\title{The robustness of a simple dynamic model of island biodiversity to geological and eustatic change}
\author{Pedro Santos Neves*, Joshua W. Lambert*, Luis Valente, Rampal S. Etienne}
\date{July 2021}

\begin{document}

\maketitle

\section*{Abstract}

Aim: Biodiversity on island is affected by various geo-physiscal processes and sea-level fluctuations. Oceanic islands (never connected to a landmass) are initially vacant with diversity accumulating via colonisation and speciation, followed by a decline as islands shrink. Continental island have species upon formation (when disconnected from the mainland) and may have transient land-bridge connections. Theoretical predictions for the effects of these geo-processes on rates of colonisation, speciation and extinction have been proposed, but methods of phylogenetic inference assume only oceanic island scenarios without accounting for island ontogeny, sea-level changes or past landmass connections. Here, we analyse to what extent ignoring geodynamics affects the inference performance of a phylogenetic island model, DAISIE, when confronted with simulated data that violate its assumptions. \\

Location: Simulation of oceanic and continental islands.

Methods: We extend the DAISIE simulation model to include: area-dependent rates of colonisation and diversification associated with island ontogeny and sea-level fluctuations, and continental island with biota present upon separation from the mainland, and shifts in rates to mimic temporary land-bridges. We quantify the error made when geo-processes are not accounted for by applying DAISIE's inference method to geodynamic simulations. \\

Results: We find that the robustness of the model to dynamic island area is high (error is small) for oceanic islands and for continental island that have been separated for a long time, suggesting that, for these types, it is possible to obtain reliable results when ignoring geodynamics. However, for continental islands that have been recently or frequently connected, robustness of DAISIE is low, and inference results should not be trusted. \\

Main conclusions: This study highlights that under a large proportion of island biogeographic geo-scenarios (oceanic islands and ancient continental fragements) a simple phylogenetic model ignoring geodynamics is empirically applicable and informative. However, recent connection to the continent cannot be ignored, requiring development of a new inference model. \\

\clearpage

\section*{Introduction}

The study of biogdiversity on islands rests upon the foundational work of MacArthur and Wilson's (1963, 1967) equilibrium theory of biodiversity (ETIB). The theory describes island diversity as determined by species colonisation and extinction, governed by the island area and isolation. The ETIB proposes that an equilibrium state of biodiversity emerges when the rates of colonisation and extinction are equal. Additionally, diversity can accumulate through \textit{in situ} speciation, particularly on large isolated islands (MacAt



\end{document}
